\chapter{Background}


\section{Energy consumption awareness in Scientific Computing}
Power consumption and energy efficiency have become of paramount important topic in 
computer science, due to the major growth bottleneck that energy has imposing to systems. This bottleneck has been hindering the further development and usage of technology such as mobile devices and low energy devices. In addition, high throughput and scientific computing have also been affected by energy consumption increase. In these fields, the large amount of data that has to be stored and processed have been increasing with the time. 

Moreover, the increasing concerns with energy consumption and its social, economical 
and environmental impact in our society has given a bigger dimension to the discussion.

The importance given to energy consumption among the research community is visible in the amount of research done in this field.

\subsection*{Literature review} %10/10
As stated by \cite{SCICOMPUTING_REQS}, scientific computing is often characterized by requiring enormous data storage capacity, high processing capabilities and complex configuration. High processing capabilities and massive data storage are requirements that potentially require massive amounts of energy. Thus, the scientific 
computing community is looking into energy efficiency with special interest.

Given its importance, many research studies have worked on technologies to increase energy efficiency and lower power consumption of computing systems. There is a vast panoply of studies with different
approaches toward improving energy efficiency. For example, to use graphic processing units (GPU) for data processing \cite{GPU}, \cite{GREENGPU}. The GPU is a specialized circuit developed to manipulate and process visuals and responsible for a fast display of complex animations based on mathematical renderings. According to \cite{GPU}, the benefits from using GPU to process data is due to its parallel architecture. The parallel architecture of GPU is well suited for intensive multimedia applications and it does the task in a low energy manner. Consequently, \cite{GPU} states the high-performance GPU achieves better results in terms of energy efficiency that any combination of CPU and GPU. They tested their results with a compute-intensive workloads. Given the rise of GPU processing in compute-intensive workloads, \cite{GREENGPU} takes an holistic approach and studies energy efficiency taking into consideration how GPU and CPU work together. They state that with the correct scheduling between both GPU and CPU, the system can save up to 20\% energy.  

Apart from using GPU for low energy data processing, there have been some attempts to leverage RISC architectures in scientific and high throughput environments. Some examples are \cite{AALTO_ARM} \cite{ACAT13ARM},
\cite{ACAT14ARMDAVID}. In \cite{AALTO_ARM}, the authors show that ARM architectures present a potential energy saving alternative when compared to Intel x86 architectures. Although, in some use cases, the an ARM based data center can become more costly than the widespread solutions using Intel. Based on the study conducted by \cite{AALTO_ARM}, the ARM clusters are financially fit when performing lightweight computational applications. On the other hand, when it comes to computation-intensive applications, the amount of ARM nodes needed to perform the tasks increases substantially. This makes the solution less interesting from a financial perspective. However, if one is only concerned with the overall energy efficiency of the cluster, ARM-based clusters show a better energy efficiency factor than Intel. In \cite{ACAT13ARM} and \cite{ACAT14ARMDAVID}, present their initials results on using ARM architecture for computing intensive applications in a scientific environment. They report how a ARMv7 based development board performs under scientific computing workloads. Their conclusions show that ARM based architectures show great potential for use of the typical scientific workload with HTC.

Several other directions have been taken toward energy efficient computing \cite{QUESTIONS_ENERGY} \cite{GREENING},
\cite{ENERGY_DILEMMA} and the number of related works keeps growing. In \cite{QUESTIONS_ENERGY}, the authors mine well-known on-line forums, such as StackOverflow \cite{STACKOVERFLOW}, to understand what are the main concerns of developers regarding application level energy efficiency. An interesting conclusion then drawn from the study is that even though the questions asked are generally interesting and relevant from a scientific point of view and correctness, the answers are generally poorly addressed. The answers have been shown to be either vague of flawed. They also contributed with a summary of software energy consumption problems that developers and end users focus the most. In \cite{GREENING}, they authors explore the potential of geographically distributed in Internet-scale systems. Finally, the authors of \cite{ENERGY_DILEMMA} compare how massive data centers compare to distributed data centers in terms of energy consumption. They conclude that distributed cloud platforms outperform the classic data center model. They also show how the MapReduce technique can be leveraged to save energy in a distributed data center model. 

\section{The European Organization for Nuclear Research and the LHC}

The European Research for Nuclear Research (CERN) \cite{CERN} is a particle physics
research laboratory sited in the Franco-Swiss border. At CERN, thousands of
engineers and physicists from about 21 state members conduct researches about the
fundamental structures of the Universe. In order to perform experiments that
support scientific research, CERN has built several particle accelerators and detectors.
The most outstanding collider is the Large Hadron Collider (LHC). The LHC
consists of a 27-kilometer ring of superconducting magnets that boost the
particles while traveling through it. The particles are accelerated in two
beams, traveling inside the LHC in opposite directions. When the beams are
traveling close to the speed of light, they are made to collide in the different
colliders. After each collision, particles and subatomic particles are projected
due to the collision, which is tracked and recorded by the colliders. The data
acquired from the collisions is then filtered and the most interesting
information is stored into the CERN data-centers for posterior reconstruction
and processing. Given the frequency of the collisions and the massive amount of
data to store and process from each collision, the LHC is an example of a scientific
computing endeavor which energy management is critical.  


\subsection*{Literature review}

According to \cite{ACAT13ARM}, the computing requirements for HPC have increased 
particularly in recent years. Most notably, a project with the magnitude and complexity of the 
LHC is a sound example of it. To achieve results like the discovery of
the Higgs boson \cite{HIGGS1} \cite{HIGGS2} and other significant scientific
advances, a massive amount of computational resources - and thus energy - was
necessary. Given the enormous amount of resources needed for storing and
processing the data, it was not viable to concentrate all the tasks in one
single super-node. Thus, the solution was to distribute the processing tasks across 
several partners and institutions through a distributed network of nodes, called
the Worldwide LHC Computing Grid (WLCG) (the same approach as studies in \cite{ENERGY_DILEMMA}). The WLCG \cite{WLCG} is a grid
computing platform where more than 170 computing centers spread across 40
countries \cite{WLCG} collaborate to store and process the data coming from the
LHC experiment. According to \cite{WLCG}, the WLCG alone is responsible for the distribution, 
storage and processing of more than 30 Peta-bytes annually. According to
\cite{ACAT13ARM}, the equivalent capacity of WLCG in 2012 was between 80,000 and 100,000 x86-64 cores, which would be difficult to build and maintain in a single super data center.

In the future, other projects and researches will demand even more processing capacity from
the WLCG. For example, as stated by \cite{ACAT13ARM}, in order to upgrade the
luminosity of the LHC detectors to its full potential, the datasets will
increase size by two to three orders of magnitude, with processing power
increasing in proportion.

These outstanding numbers, in addition to the price of energy and the increasing
concerns with green computing, highlight the importance of
developing more efficient and methods and techniques high performance computing,
both in the scientific computing in general and in the LHC computing grid in
particular. 

There are several studies concerning computational energy efficiency at CERN. In \cite{INTEL_WP}, Intel researchers present methods to better use Intel hardware for energy efficient computing at CERN. The white paper also presents some future solutions based on the x86 technology developed by Intel that will eventually perform better from an energy point of view than the current solutions. In \cite{IGPROF}, the authors describe how they developed a energy profiling module on top of IgProf to better understand how energy is consumed by the system. The authors focused primarily on running the energy profiling on IgProf with tasks used by the software of LHC. The starting point of \cite{IGPROF} is marked by stating that high throughput computing in general - and scientific computing at CERN in particular - has to rely on tools that provide a deep understanding of how power is used in order to develop energy efficient solutions.

In \cite{FUTURE_CERN}, the authors conducted a survey on the potential advantages of using x86-64 variants, ARMv7 32-bit, ARMv8 64-bit, Many-Core and GPU solutions in the LHC context. The main focus was to evaluate whether the solutions surveyed would present advantages from an energy efficiency perspective. In \cite{POWER_AWAR_CERN}, the authors conducted a research on how both power-aware software applications and scheduling algorithms could improve energy efficiency in both distributed and centralized data center models. The authors use the LHC as a powerful distributed processing data center to access the results. In \cite{EXPLORATIONS_CERN}, the authors conducted a research on the viability of ARM processor and the Intel Xeon Phi co-processor for scientific computing in the CERN context.

\section{Tools and techniques for measuring energy consumption} %5 (new articles)

Many research works have discussed the importance of the right tools and techniques for measuring the energy consumption of computing systems. As \cite{CYCLE_ENERGY}, \cite{QUESTIONS_ENERGY}, \cite{ACAT14ARMDAVID}, \cite{ACAT13ARM},\cite{ACAT}, \cite{WIRELESS_ENERGY} and other studies mention, accurate techniques and tools for measuring power consumption are of paramount importance to understand the bottleneck components that need to be improved from a energy consumption point of views. Without accurate measurements and feedback from the computing systems, it becomes difficult to improve the energy performance in a larger scale. As stated by  \cite{QUESTIONS_ENERGY}, \cite{CYCLE_ENERGY} and \cite{ACAT}, there is few research studies on tools and techniques for energy measurement of computing systems at the different levels.   


\subsection*{Literature review}
The study conducted by \cite{QUESTIONS_ENERGY}, shows that engineers have been
considering energy consumption as an important factor when developing software.
The research consists of an empirical study that aims to understand the opinions and
problems of software developers about energy efficiency. The data that sustains
the conclusions are  taken from
a well-known technical forum, (\textit{StackOverflow} \cite{STACKOVERFLOW}).
Although the study is focused in an application-level energy efficiency, it
shows that developers are aware of the importance of energy efficiency in 
computational systems. When trying to understand in depth what questions arise 
more frequently, it is shown that measurement techniques is amongst the most
asked questions by developers. In addition, the study ascertains that the 
\textit{"lack of tool support"} is an important handicap for the development of 
energy efficient software.

The authors of \cite{CYCLE_ENERGY}, recognize that the accurate understanding of how energy is measured in computing systems is the basis for any high level decision on how to improve the energy efficiency. In addition, the study states that conventional measurement techniques have serious limitations. Therefore, they introduce a new methodology to measure energy consumed by the system on the chip, which is capable of sampling energy consumption in real time. They used a ARMv7 processor to test their measurement technique, which is shown to be accurate and of interesting applicability when measuring energy consumed by computational systems.

Other research works such as \cite{ENERGY_ALGORITHMS}, \cite{MODEL1} and \cite{MODEL2} leverage mathematical models and algorithms to represent the energy consumption of complex computing systems and data centers. Even though we do not undertake this approach in our study, these studies highlight the necessity of proper understanding how the computing systems spend energy in order for researchers to be able improve energy efficiency. In \cite{MODEL1}, the authors claim that energy models are pivotal in design and optimization of energy efficient data centers. They surveyed the state of the art models for prediction of energy efficiency in data centers. One interesting finding of this study is that many of the models existing to date take into consideration only the CPU related energy consumption, rather than the overall system power consumption of the data centers. This fact, the authors claim, leads to less accurate energy consumption mathematical models of data centers. 



\section{RISC in scientific computing} %6 (new articles)

One possible approach to increase the energy performance of HTC and scientific computing data centers is by using low energy processing units. The Reduced Instruction Set Computing (RISC) architectures have been developed and used to minimize energy consumption on mobile and energy constrained devices. An open question among the research community is whether RISC architectures are capable of performing well under the computing-intensive workload that HTC and scientific computing require. One concrete example of a widely used RISC architectures are developed and produced by ARM \cite{ARM}.

The ARM architecture \cite{ARM} is a RISC architecture developed by the company ARM Holdings. As a RISC-based chip, the ARM processors require less transistors than the typical CISC solutions. According to \cite{CISC_RISC}, this approach reduces the overall power consumption of the RISC architectures when compared to CISC architectures. For this reason, ARM units have been widely used in low-energy power devices such as embedded systems, laptops and smart-phones. These applications are responsible for the rise of ARM processors in the market in the last decade. The 32 bit ARM architecture, mostly known for ARMv7, is one of the most popular architectures in the mobile devices. While ARMv7 has been taking over the low-energy processor market, the ARMv8 architecture \cite{ARMv8} introduces the a RISC 64-bit with focus on power efficient devices. 
Given the fact that ARM architecture were developed from the ground up with energy efficiency as main priority, studies such as \cite{ACAT14ARMDAVID}, \cite{ACAT13ARM} and this thesis have delved around the idea of using the ARM architectures to improve the energy efficiency of HTC data centers. 

\subsection*{Literature review}

Several studies have been researching on how can RISC architectures in general - and ARM in particular - be used to improve energy efficiency. In \cite{CISC_RISC}, the authors compare ARM-processor based clusters with Intel clusters from a energy efficiency and cost efficiency perspective. They analyzed and compared ARM Cortex-A8 and Cortex-A9 with Intel Atom and Intel Sandy bridge microprocessors and concluded that there is no fundamental difference in the design of CISC and RISC microprocessors that could make one more energy efficient than the other. There is, instead, different design considerations when Intel and ARM developed their processors and that is what makes the difference in terms of energy efficiency. Another study on the comparison between RISC and CISC architectures in terms of power consumption is \cite{CISC_RISC_2}. In \cite{CISC_RISC_2}, the authors acknowledge that the once marginally used RISC architectures are blooming, due to its use in low energy devices such as mobile devices and embedded systems. Furthermore, they also state that the traditionally low power ARM architecture is entering in the high-performance server market. As in \cite{CISC_RISC}, \cite{CISC_RISC_2} conclude that the energy performance edge of RISC over CISC architectures exists because of design points, rather than intrinsic characteristics of either RISC or CISC. The authors conclude, thus, that the size of the instructions set in the CPU does not account for the energy performance of the cores. Indeed, the discussion among the research community about the differences between CISC and RISC architectures in performance has been happening for a while. In 2009, the authors of \cite{CR_TALE} yet again compared CISC and RISC from a performance perspective.  

In \cite{RISC_PERF}, the authors acknowledge that to process HTC tasks in RISC architectures is a challenging task due to its design particularities that inhibits performance. The research conducted by \cite{EVAL_ARM} aims at studying several characteristics of ARM processors such as energy efficiency, execution time and power consumption. Their main goal was to verify whether clusters using RISC architectures - and thus with low power consumption - would be feasible. The experiments conducted during the study used an ARM development board, as in our study. The authors of \cite{EVAL_ARM} concluded that ARM clusters show potential as an alternative to the widely used Intel x86 architectures.

Even from a broader point of view, RISC architectures are been seen as an alternative for the widely used CISC architectures. In \cite{RISC_CASE}, the authors claim that the future generation of very large scale integration (VLSI) computers might be more effectively implemented as RISC than as a CISC architecture.

In \cite{ACAT14ARMDAVID}, a server-purpose ARM machine is compared with the recent
Intel architectures, such as the recent Intel Xeon Phi and a dominating Intel product
intended for HPC workloads (Intel Xeon E5-2650). The workload for comparing
the architectures was ParfullCMS. They based the results on performance (events per
second) and scalability over power (watts). In addition to performance and energy
consumption comparisons, the paper describes the porting endeavors of the CMSSW to
an ARMv8 64-bits architecture.

In \cite{ACAT14ARMDAVID}, they use an APM X-Gene 1 running on a development board. 
It consists of a 8 physical core processor running at 2.4GHz with 16GB DDR3 memory.
As the authors highlight, the firmware for managing processor ACPI power states was
not yet available when the study was made. Thus, it is expected that the energy 
performance will improve once the firmware is available \cite{ACAT14ARMDAVID}.

Under the circumstances of the experiment, the overall results show that APM X-Gene 
is 2.73 slower than Intel Xeon Phi. From the energy consumption performance (events
per second per watt), the Intel Xeon E-2650 is the most efficient, with APM X-Gene
presenting similar performances despite the absence of platform specific 
optimizations. Therefore, \cite{ACAT14ARMDAVID} concludes by stating that the APM
X-Gene 1 Server-On-Chip ARMv8 64-bit solution is relevant and potentially interesting
platform for heterogeneous high-density computing. 

According to \cite{ACAT13ARM}, processors have hit scaling limits after the year of 2015. There has been a development of multiprocessor architectures that
allow to run parallel tasks and the time clock frequency has stabilized. Most High Physics Computing systems run in clusters of several cores. According to \cite{ACAT13ARM}, additional cores are parallelized and can run at the same time, which allows the system to scale. However, also commodities such memory, I/O streams and energy
scale proportionally in such architectures.




\section{Workload scheduling based on dynamic energy pricing}

The workload scheduling based on dynamic energy pricing can be a good approach when certain conditions are in place. Firstly, in situations when the energy price changes according to the overall power grid energy usage. Secondly, when there are machines with different energy efficiency and computing performance available. Given these conditions, the main concept is to schedule the workload to more energy efficient machines when the energy price is higher. On the other hand, when the energy price is lower, the more performing machines can be used without huge degradation of the overall price budget.


\subsection*{Literature review}

There are several studies exploring inter data center solutions to lower
the electricity bill by leveraging the spacial-time dynamic of energy pricing. The 
emphasis is given to job scheduling across data centers that are located in 
different places. The main idea is to exploit the fact that energy prices 
are change based on location and time. The research community is mostly concerned 
with fairness, server availability, queue delays, bandwidth costs with job migration
and quality of service.. In addition there are several research studies related with 
migration of cloud computing jobs. Studies that in one way on another address this 
perspective are \cite{EFF_JOB_SCHEDULING}, \cite{MIGRATION_CLOUD}, 
\cite{MINIMIZING_DIST}, \cite{CUTTING_BILL}, \cite{SCHED_HETEROGE}, amongst others. 

Besides inter center solutions, the research community has been addressing the power 
consumption of the computing nodes specifically from a data center perspective.
This perspective is closer to what we are trying to achieve with our solution.
For example, one work that seems closer to our solution is \cite{DYN_PRICING_HPC}.
In this study, the authors achieve better energy performance in a dynamic
pricing environment with HPC systems by judiciously scheduling parallel jobs -
which have different energy profiles - depending on the energy pricing of the
moment. The main difference to out solution is that the performance of the machines 
are not taken into consideration when scheduling the jobs, but rather the job energy 
profiling.

Another research study that related to our solution is \cite{TASK_SCHED}. They
came up with an optimal algorithm and two heuristic algorithms to schedule tasks
to heterogeneous processors. In addition, they also take into consideration the 
memory allocation in heterogeneous memory in order
to minimize energy consumption while meeting the assumed deadlines. Their work,
though, seems to go further than our solution since it considers heterogeneous
memory allocation as well. They consider is a
computing process executing several tasks in a parallel computing environment. The
system consists in a variety of different computational node, each of one with a
given number of processors. All computational nodes are connected by a
high-speed network. Thus, all  the processors can cooperate and realize complementary
and parallel tasks. From the energy point of view, the processors of each 
computational node have an energy profile assigned and have a certain frequency, 
which will be taken into consideration when scheduling the task. The work dates 
from end of 2014, which indicates that this is a trendy and hot subject, but it 
seems that our approach is been used already. 

A similar idea has been explored in \cite{EXE_METHOD}. They present only
heuristic algorithms to schedule tasks on heterogeneous computing systems,
based on efficiency and energy consumption. They develop heuristic algorithms
due to the fact that an optimal solution for the needed scheduling is
NP-complete.  


According to \cite{EFF_JOB_SCHEDULING}, because of the magnitude of energy costs in data centers, it is important to
lower the energy consumption in data centers. The servers are composed of
heterogeneous machines from the performance and energy efficiency. In addition,
the data centers may be disposed in different geographical locations and, thus,
have different energy tariffs. The authors of \cite{EFF_JOB_SCHEDULING}, claim that
the key idea to lower the energy bill in data centers is to have energy
efficiency servers and schedule the jobs to where energy is more affordable at a
given time. In the context of servers distributed over different geographical locations, it
is also important to satisfy fairness and delay constraints. This scenario is
less critical when the server is not distributed, as in our case.

In \cite{EFF_JOB_SCHEDULING}, the authors present an on-line scheduler that
distributes batch workloads across multiple data centers geographically
distributed. The scheduler aims to minimize the energy consumption of the set of
servers having into consideration fairness and delay requirements.  The scheduler is inspired on the technique that
optimized time-varying systems. The algorithm takes a queue of jobs schedule them to the different servers
having in consideration the (1) server availability, (2) energy price and (3)
job fairness distribution. Consequently, the algorithm is tuned to calculate the
trade-off between energy pricing, fairness and queuing delay.

In \cite{EFF_JOB_SCHEDULING}, the scheduler developed takes into consideration
the server availability, energy costs, fairness and queuing delay to schedule
random jobs arrivals. It opportunistically schedules jobs when (and to where) 
energy prices are low. Comparing to our study, though, we do not consider geographically distributed 
servers but rather, we have schedule the jobs based on the heterogeneous set of 
machines existing on the server.


The \cite{MIGRATION_CLOUD} study aims to exploit the temporal and geographical variation of
electricity prices, in the context of data centers. They study algorithms to
schedule (migrate) jobs in data center based on the energy cost and
availability. When the servers are in different geographical location, costs with data
migration have to be taken into consideration, namely bandwidth costs of moving
the application state and data between data centers. The bandwidth costs
increase proportional to the amount of data migrated between servers. Their study focuses on inter data center optimization, rather than intra data
center optimization. The algorithm differs from others in 3 major differences: First, they consider
migration of batches of jobs. Second, the algorithm has into consideration the
future influence of the job scheduling, providing robustness against any future
deviations of the energy price. Finally, they also take into consideration the 
bandwidth costs associated with job migration across data servers.The main point is to provide a good trade off between the energy pricing and the
job migration, taking into consideration the bandwidth prices. Comparing to our study, we do not approach the problem from an inter data center
perspective, but rather from an intra data center, by scheduling the jobs to
machines depending on their energy performance and the actual energy prices. One
interesting idea from this study that can be used, is the usage of an online
algorithm that takes into consideration the expected prices and also the actual
prices.


In \cite{MINIMIZING_DIST}, the researchers try to systematically study the problems of how
minimize the electricity cost in data centers while guaranteeing minimal quality
of service. To that end, they take into consideration the local and time diversity 
of electricity prices. The contributions are twofold: In one hand, they show that local and time
dependent electricity pricing can be leveraged to minimize total energy price
of clusters of data centers. On the other hand, they present a mixed-integer
optimization formula with linear programming formulation to show that the energy
pricing of clustered data centers can be improved under such conditions. To model the total of electricity costs, they assume that all the servers have a
similar power profile - which means that all the servers, disregarding their
locations, have the same workload. They calculate the power consumed by the
server by multiplying the total of servers at a certain region by the total of
workload they have.  They obtain the most efficient solution, they approximate an optimization
problem through a linear programming formulation and then, convert the linear
programming formulation to a minimum cost flow problem. 

This work dates from 2010 and don't take into consideration the bandwidth costs
of migrating the batches between data centers. Even though that is not an issue
in our study, this is taken into consideration in other works such as
\cite{MIGRATION_CLOUD}. Again, it is part of the set of studies on inter
data-center and electrical costs optimizations that location and time based 
pricing allows.


The authors of \cite{CUTTING_BILL} show that existing systems may be able to
save millions of dollars by judiciously schedule workload to servers taking into
consideration the temporal and geographical variation of energy prices. The
results are based in historical data collected on Akamai's CDN. 

In \cite{DYN_PRICING_HPC}, the authors leverage the fact that parallel jobs have 
distinct energy profiles. Taking it into consideration, they study the impact of
scheduling jobs according to the energy prices at a given moment and the job's
energy profiles. So, the study aims to reduce the electricity bill by scheduling and
dispatching jobs according to their energy profile. Their solution has a
negligible impact on the system's utilization and scheduling fairness. 

Their basic idea is to schedule jobs with low energy profile during on-peak
electricity time and, on the other hand, schedule jobs with high energy profile
during the off-peak electricity time. In addition, the scheduling is done in
such a way that it is guaranteed that there is no degradation of the overall
system performance. 

The authors claim that "A key challenge in HPC scheduling is that system 
utilization should not be impacted. HPC systems require a tremendous capital 
investment, hence taking full advantage of this expensive resources is of great 
importance to HPC centers.". This may make impractical and wreck our solution,
because of the inevitability of turning off (or idle) great amounts of computing 
resources. Although, Internet data centers (cloud data centers)  may be a good 
match to our solution: usually there are much less resources being used at a given 
time than in HTC computing [need confirmation, partially mentioned in this article]. The scheduling algorithm used places jobs in a time-window. The jobs are chosen
to run based on job fairness, job energy profile and energy prices at a given
time. A greedy algorithm and 0-1 Knapsack based policy are used to minimize the
electrical costs. Their results show that gains in the order of 23\% can be obtained without
impact on the overall system. 

According to the survey carried by \cite{DYN_PRICING_HPC}, the dynamic energy
pricing has been implemented in the biggest markets in Europe, North America,
Oceania and China, while Japan was at the time starting to test it on its major
cities. They develop two power aware job policies: 1) greedy approach, where jobs are
allocated based on their energy profiles  and 2) 0-1 Knapsack based policy,
where both job profile and system utilization are taking into consideration.

The authors of \cite{SCHED_HETEROGE} present a novel task scheduling algorithm
for HPC systems which considers two main points: reducing the energy consumption
of the overall system and minimize the schedule length. An HP system is defined
by the authors as set of distributed computing machines with different
configurations connected through a high speed link to compute parallel
applications.

They assume a that all the information needed to schedule the task is known
beforehand. The scheduling algorithm assigns then the jobs to the different
machines. Thus, the scheduling algorithm is said to be static, in opposition to,
for example, the on-line algorithms.  One of the particularities of the algorithm is to reduce the impact of
duplication-based algorithms. The duplication-based algorithms schedule jobs
across machines redundantly, in order to maximize performance by eliminating
intercommunication between tasks. However, from the energy consumption point of
view, it is not th ideal situation since more than one processor are performing
the same job. Once again, this research work aims at improve the energy efficiency of HPC
systems at a distributed level and do not focus, as our approach, on inter data
center solutions.

 In \cite{TASK_SCHED}, the authors address the problem of an energy aware
scheduling for heterogeneous data allocation and task scheduling. The problem
consists in finding the best tasks scheduling in a heterogeneous system that meet
the deadlines while minimizing the energy consumption.

The processors and memories come in different flavors nowadays in HPC
systems, making complex the task of efficiently schedule processor power and 
memory space in an energy efficient way. The problem of finding an optimal
processor and data scheduling becomes critical when trying to minimize energy 
consumption and meet imposed deadlines. As the study shows, there are several research efforts tackling the task
scheduling problems on heterogeneous computing and, most notably for our
research, \cite{EFF_DSP}.They present an optimal algorithm and two heuristic algorithms to solve the
HDATS problem, since the optimal algorithm takes too long to solve problems
until 100 nodes. The optimal solution has two phases: First is uses the
DFG\_Assign\_CP algorithm to better map each task to node. Secondly, it choses the
data assignment to whose total energy consumed is reduced and the deadlines met.

The authors of \cite{EXE_METHOD} claim that, unfortunately, there are not many
studies of processor scheduling algorithms that take into consideration both
time and energy. In this study, they explore heuristic scheduling algorithms
focused on high performance computing and green computing. They work on
heuristic algorithms and not in the optimal algorithm, because the optimal
algorithm is proven to be NP-complete. 
