\chapter{Conclusions}

Energy consumption has has become a major bottleneck in HTC and scientific computing, where the amount of data to analyse has been increasing manyfold every year. Besides the economical issues of the increasing energy consumption, social and environmental concerns should be also considered. Therefore, to research how to build and develop energy efficient HTC systems has become of paramount importance among the research community.

The goal of this research was to understand wheter RISC architectures are capable of improving the energy efficiency of scientific computing without performance degradation, when compared with the current x86 Intel architectures.

In order to achieve that goal, we conducted reasearch on tools and techniques for measuring energy consumption at different system levels. In addition, we conducted experiments that aimed at comparing the energy performance of ARM and Intel architectures, working under real world workloads and frameworks. Finally, we used the results of the experiments to develop a scheaduling algorithm that optimizes the eletricity bill of heterogeneous data centers working in a dynamic eletricity pricing markets. 

\vspace{5mm}

Our main contribuitions are: 

\begin{itemize}
  \item We researched and outlined best practives for different system levels. The results of this work were published in the conference proceedings of the 16th International workshop on Advanced Computing and Analysis Techniques in physics research (ACAT'2014). The article was called \textit{Techniques and tools for measuring energy efficiency of scientific software applications}

  \item Our experiments and research shown that ARM architecture shows potential for an energy savings in HTC when compared to the x86 systems widely used nowadays. 

  \item Our research shown that heterogeneous computing can be leveraged in HTC in markets where the price of the eletricity is dynamic. We shown how to achieve savings in such environment and developed a scheaduling algorithm that accomplishes that.
\end{itemize}
