\chapter{Analysis}



The scope of the analysis presented in this section is twofold: to compare the
platforms from an energy efficiency perspective and analyze the tools and techniques used on the different experiment sets. 

Whereas the first two sections analyze the energy efficiency of the platforms
studied and the particularities of the tools and techniques used, the last 
section covers the results and issues which arose when using RAPL to measure 
the energy consumption in a NUMA environment.

\section{First Set of Experiments}
ARM server vs ATOM and QUAD, using clap and software-based experiments

In figures~\ref{fig:aalto_quad_clamp}, ~\ref{fig:aalto_atom_clamp} 
and~\ref{fig:aalto_arm_clamp}, it is plotted the physical measurements from the
beginning of the workload until the end.

\begin{description}
\item[Stages] \hfill \\
All the experiment sets show 3 stages. The stages can be better identified 
when plotting
the memory workload against cpu usage, rather than the energy consumption 
measurements (see Figures in GDrive-Add?). The three stages consist in different 
phase of the
experiment. The first stage consists on the initialization process. During this
stage mostly memory is being used, rather than cpu workload. The second stage
is the connection phase. It has the goal of fetching the  meta data fetching from
the CERN servers needed to perform the reconstruction of the events. Anew, 
during this stage, the cpu load is low when compared to the 
memory workload. Lastly, the third stage corresponds to the event processing
phase. Therefore, the last stage is cpu intensive and the one that is performing 
the useful computation for the reconstruction of events.

\item[Stages comparison] \hfill \\
Regardless the number of processes running, the time for the three stages is 
constant in all the experiment sets, if the cpu is not overcommitted. When
the number of processes exceed the number of available cores, the time to 
process the events increases since there are no available cores to process the
events concurrently. In the overcommitted situation, the time increase follows
a ratio \textit{nr\_of\_processes/nr\_of\_cores\_available}. For example, if the
number of processes running is 6 and the number of cores available is 4, the
time needed to process the events increases roughly 2/3 compared to when the
cpu is not overcommitted.   


\item[Importance of the stages] \hfill \\
Unarguably, the most important stage when studying the energy efficiency of
workload in CERN is the third stage. There are two main reasons for that: first,
 the CMSSW configuration at either CERN, 2nd and 3rd tiers has proxies
and caches that speedup the second stage [refs]. Lastly, given the amount of
data to be processed in the last phase and thus the energy consumed by the
event processing stage, the energy consumed by the former stages becomes
 irrelevant. Therefore, in the remainder of the chapter we focus our analysis on 
 the event processing stage only. The energy measurements of the third stage are
shown in the figures
~\ref{fig:aalto_quad_events},~\ref{fig:aalto_atom_events} and~\ref{fig:aalto_arm_events}. 




\end{description}


\subsection{Comparison ARM and Intel architecures}
\subsection{Tools and techniques}

\section{Second Set of Experiments}
ARM board and Intel Xeon, using on chip and external measurements

\subsection{Comparison ARM and Intel architecures}
\subsection{Tools and techniques}


\section{Third Set of Experiments}
Intel Xeon, using RAPL to measure energy consumed by the different nodes, with
different types of binding
