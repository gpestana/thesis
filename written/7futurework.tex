\chapter{Future Work}

From the experiments perspective, we believe that it would be valuable to run more experiments in environments closer to production than development boards. The output of such experiments would present valuable complementary data to what we have got. Another interesting research work would be to understand if the drawbacks of the approach make it inviable in a real production scenario.


In addition, to run more experiments using a methodology where the tools and techniques are as accurate as possible (based on the learning of Chapter 3) and where the results can be compared across all the experiments. This might be difficult to achieve given the different tools available to measure in the different platforms. Another interesting research subject would be to develop a cross platform and accurate way to perform high resolution power consumption measurements.


It would be interesting to develop further the scheaduling algorithm for dynamic eletricity markets and HTC presented in the Chapter 6. In addition, to apply the same idea to different well known scheduling algorithms. Also, to improve the energy model used and increase the complexity of the energy profiles in order to the results to be as close to the reality as possible. Another interesting research work would be to understand if the drawbacks of the approach taken make it inviable to use in a real production scenario.

