\chapter{Experiments}


During our research, we have performed several experiments under different
settings. The main goal was to understand how the ARM and Intel architectures
perform under authentic scientific computing workloads from an energy
consumption standpoint. Furthermore, our goal was to compare the results
obtained to evaluate the potential of ARM architectures to perform HPC tasks, in
comparison to the Intel architectures.
The software used to run or simulate authentic scientific computing tasks is
widely used in production and research at the CMS experience. We have used the
CMSSW framework [ref] (see section Y below) and ParFullCMS [ref] (see section X
below).  

We organized the experiments in 3 sets. The sets differ from methodology,
hardware and software used and general experimental conditions. However, the
main goal of such experiments is always the same: to understand if ARM
architectures have potential to replace Intel in scientific computing from an
energetic standpoint. The techniques and tools used to perform the energy
consumption measurements were based on the study presented on the previous
chapter. The setups of the experiments and tools used to perform the energy
measurements during the experiments are explained and detailed on section W.

The remainder of this chapter in two main section. Firstly, we will roughly describe the
architectures of the hardware used during the experiments and the reason why we
chose them. Secondly, we describe
the setup of the experiments. The former section is organized as per set of experiment. 
For each setup, we present the hardware and software setups and energy measurement tools
used. We will refer the batch of experiments as first (1SE), second (2SE) and
third (3SE) set of experiments throughout the rest of the document. 


\section{Hardware}
The focus of this work is given to the comparison of ARM architectures with x86 based 
processors. The choice was also conditioned to the machine availability when this
study was conducted. The ARM architecture based machines we used were a single-board 
ARM processor developed by Odroid \cite{ODROID_XU3} and a server class ARM processor by Boston 
Viridis \cite{VIRIDIS}. The x86 machines used were Intel Sandy Bridges and Intel Bonnell
microarchitecture.


\subsection{ARM architecture}
\subsubsection{Boston Viridis server}

The Boston Viridis server is one of the first ARM architecture based servers
where the processors, IO and networking are fully integrated in one single chip.
According to \cite{VIRIDIS}, the server is intended to work in web servers,
cloud and data analytics environment with outstanding power performance
\cite{VIRIDIS}.

\vspace{5mm}

\begin{figure}[h!]
  \centering
    \includegraphics[width=\textwidth]{"img/viridis&SoC"}
    \caption{A. Viridis Server chassis with 12 energy card in it. B. Energy card
with 4 nodes. Taken from \cite{VIRIDIS}}
    \label{fig:viridis&SoC}
\end{figure}

\vspace{5mm}


The Boston Viridis server consists of a chassis with twelve racks for energy
card and each energy card contains four nodes \ref{fig:viridis&SoC}. The nodes are ARM based cores
fabricated by Calxeda. The block diagram of a Viridis node shows the
architectures of the ARM SoC. Each node contains four ARM A9 Cortex core CPU up
to 1.4Hz. A memory controller and L2 cache can also be found in the SoC. In
addition, a couple of energy management blocks and IO controllers complete the
Calxeda EnergyCore processor \ref{fig:calxedaSOC}.


\begin{figure}[h!]
  \centering
    \includegraphics[width=75mm]{"img/calxedaSOC"}
    \caption{Block diagram of Calxeda EnergyCore. Taken from \cite{VIRIDIS}}
    \label{fig:calxedaSOC}
\end{figure}

According to \cite{CORTEXA9}, the ARM A9 Cortex is a popular and mature general
purpose core for low-power devices. It was introduced in 2008 and it remains a
popular choice in smartphones and applications enabling the Internet of Things
(IoT) \cite{CORTEXA9}. The ARM A9 Cortex supports a widely supports the ARMv7A 
instruction set architecture. For more detailed specifications, refer to
\cite{CORTEXA9}.

\begin{figure}[h!]
  \centering
    \includegraphics[width=75mm]{"img/cortexA9"}
    \caption{Block diagram of Cortex A9. Taken from \cite{CORTEXA9}}
    \label{fig:cortexA9}
\end{figure}


\subsubsection{ODROID-XU3 development board}

The ODROID-XU3 \cite{ODROID_XU3} is an open-source development board produced by
ODROID. They claim that the ODROID-XU3 is a "new generation of computing device
with more powerful, more energy efficient hardware and smaller form factor"
\cite{ODROID_XU3}. It is mostly used for testing and platform development,
rather than used in production scenarios.

 \begin{figure}[h!]
  \centering
    \includegraphics[width=75mm]{"img/odroid"}
    \caption{ODROID-XU3 development board. Taken from \cite{ODROID_XU3}}
    \label{fig:odroid}
\end{figure}

The ODROID-XU3 processor has four Samsung Exynos-5422 Cortex A15 and four Cortex A7 cores,
with 2GB of LPDDR2 RAM. Only four cores are working at the same time and they are 
scheduled based on the big.LITTLE technology. The big.LITTLE technology
\cite{biglittle} automatically schedules workloads across cores based on
performance and energy needs. The developers claim that it can save from 40\% to 
75\% of the CPU energy, depending on the performance scenario \cite{biglittle}.
Even though the CPU contains eight cores, only four of then are working at a
given point. The block diagram of the ODROID-XU3 can be seen in 
\ref{fig:odroidxu3-diagram}.
The ODROID-XU3 has a Texas Instrument power monitor chip (TI INA231) embedded
from origin. The TI INA231 allows the users to read the energy consumed by the 
cores and DRMA at a 
sampling rate of microseconds. These readings can be easily triggered and read
through software and are a handy and accurate way to make fine-grained energy
consumption measurements.



\vspace{5mm}

\begin{figure}[h!]
  \centering
    \includegraphics[width=\textwidth]{"img/odroidxu3-diagram"}
    \caption{ODROID-XU3 block diagram. Taken from \cite{ODROID_XU3}}
    \label{fig:odroidxu3-diagram}
\end{figure}

\vspace{5mm}



\subsection{Intel x86 architecture}

Across the different experiments, we have used three different machines running
on top of x86 Intel instruction sets to
compare with the ARM based machines. The Intel x86 machines are the most wide
spread for server and workstation applications.
 

The Intel Xeon that we used had RAPL enable, which allowed us to measure energy
consumption accurately at a fine-grained level. Since the ATOM and QUAD machines
did not have RAPL technology enabled, we used a clamp power meter to measure the
energy consumed at a given time (see more information about RAPL
and energy consumption measurements in section X).


\subsection{Summary}
- Intel are more powerful, with energy consumption as a cost.


\section{Experiments setup}
\subsection{First set of experiments}

\begin{figure}[h!]
  \centering
    \includegraphics[width=150mm]{"img/1se_specs"}
    \caption{Specifications of the machines used in the 1SE}
    \label{fig:1se_specs}
\end{figure}


\subsection{Second set of experiments}


\begin{figure}[h!]
  \centering
    \includegraphics[width=150mm]{"img/2se_specs"}
    \caption{Specifications of the machines used in the 2SE}
    \label{fig:2se_specs}
\end{figure}


\subsection{Third set of experiments}

\begin{figure}[h!]
  \centering
    \includegraphics[width=150mm]{"img/3se_specs"}
    \caption{Specifications of the machine used in the 3SE}
    \label{fig:3se_specs}
\end{figure}



