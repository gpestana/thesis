\chapter{Experiments}


We performed experiments with different hardware setups. The experiments consisted on running simulations of HPC workload while measuring the energy consumed by CPU and by the whole machine.
The main goal is to compare the energy efficiency of ARM and Intel architectures. To attain that goal, we compared the results
of the experiments to evaluate the potential of ARM architectures to perform HPC tasks, in
comparison to the Intel architectures.

The software used to run the computing tasks
widely used in production and research at the CMS experiment. In order the results to be as realistic as possible, we used the CMSSW framework [ref] and ParFullCMS [ref] simulations, which are widely used in production at CERN.

We organized the experiments in 2 sets. The conditions under which the
experiments were conducted were similar. Due to hardware and software
limitations and availability, it was not possible to completely reproduce the experiment
conditions across all the sets. However, we believe that the differences will
affect the final results only to a resonable degree, making it possible to scientifically compare the results. This and other considerations will be discussed further in the Analysis chapter. 

The tools and techniques used to perform the energy
consumption measurements were based on the study presented on the previous
chapter. The setups of the experiments, methodology and tools used to perform the energy
measurements during the experiments are explained and detailed in the following
sections.

Thoughout this chapter, we will label \textit{set of experiments} as experiments conducted with
the same hardware and software configuration. The degrees of freedom of each
experiment are the number of events and number of threads processing the workload.
 
For each setup, we outline the hardware, software setups and the used energy measurement tools. 
During this chapter and throughout the rest of the thesis, we will describe
each batch of experiments as first set of experiments (1SE) and second set of experiments (2SE).

The remainder of this chapter divided in two section. Firstly, we will outline the most relevant characteristics of the
architectures used during the experiments.  Secondly, we describe
the setup of the experiments and methodology used to perform the experiments. 

\section{Hardware}
The focus of this work is to compare energy efficiency of ARM architectures and Intel based 
processors under similar workload. Our hardware choice  was conditioned to the
machine availability when the study was conducted. In addition, we also aimed
at comparing similar conditions and workloads across all the set of experiments. 

The ARM machines used were a single-board ARM processor developed by Odroid
\cite{ODROID_XU3} and a server class ARM processor by Boston 
Viridis \cite{VIRIDIS}. The Intel machines used were part of the microarchitectures family Sandy Bridges and Intel Bonnell. In the following sections, we will describe the hardware architecture, features of the hardware used to run the experiments and where the hardware is commonly used outside the scope of this study. 

\subsection{ARM architecture}
\subsubsection*{Boston Viridis server}

The Boston Viridis server is one of the first ARM architecture based servers
where the processors, IO and networking are fully integrated in one single chip.
According to the vendor, the server is intended to perform in a web server,
cloud and data analytics environment with outstanding power performance \cite{VIRIDIS}.

\begin{figure}[h!]
  \centering
    \includegraphics[width=\textwidth]{"img/viridis&SoC"}
    \caption{A. Viridis Server chassis with 12 energy card in it. B. Energy card
with 4 nodes. Taken from \cite{VIRIDIS}}
    \label{fig:viridis&SoC}
\end{figure}


The Boston Viridis server used in this study (which we will label as ARM\_viridis throughout the rest of the document) consists of a chassis with twelve racks, each with an
energy card. Each energy card contains four nodes \ref{fig:viridis&SoC}. 
A node is an ARM based CPU fabricated by Calxeda. The block diagram of a ARM\_viridis node is represented on \ref{fig:calxedaSOC}shows the architectures of the EnergyCore used. We can notice that the SoC has an energy management engine that will further on allow us to sample the energy consumed by the node.  

Each ARM\_viridis node contains four ARM A9 Cortex core with a clock speed up
to 1.4MHz. A memory controller and L2 cache is includede on the chip. In
addition, a couple of energy management blocks and IO controllers complete the
Calxeda EnergyCore processor \ref{fig:calxedaSOC}. These energy measurement
blocks were used to perform part of the energy measurements with this setup.


\begin{figure}[h!]
  \centering
    \includegraphics[width=75mm]{"img/calxedaSOC"}
    \caption{Block diagram of Calxeda EnergyCore. Taken from \cite{VIRIDIS}}
    \label{fig:calxedaSOC}
\end{figure}


According to \cite{CORTEXA9}, the ARM A9 Cortex is a popular and mature general
purpose core for low-power devices. It was introduced in 2008 and it remains a
popular choice in smartphones and applications enabling the Internet of Things
(IoT) \cite{CORTEXA9}. The ARM A9 Cortex supports the ARMv7A 
instruction set architecture. A detailed study of the ARMv7A internals is out
of scope of this work. More detailed specifications about the
internals of the ARMv7A instructions set can be found in \cite{CORTEXA9}.


\begin{figure}[h!]
  \centering
    \includegraphics[width=75mm]{"img/cortexA9"}
    \caption{Block diagram of Cortex A9. Taken from \cite{CORTEXA9}}
    \label{fig:cortexA9}
\end{figure}


\subsubsection*{ODROID-XU3 development board}

The ODROID-XU3 \cite{ODROID_XU3} is an open-source development board produced by
 Hardkernel. They claim that the ODROID-XU3 is a "new generation of computing device
with more powerful, more energy efficient hardware and smaller form factor"
\cite{ODROID_XU3}. At the time of these experiments, the ODROID-XU3 was  mostly used for testing and platform development and it was not intended to run in production scenarios. Throughout this document, the ODROID-XU3 described in this section will be called ARM\_odroid.


 \begin{figure}[h!]
  \centering
    \includegraphics[width=75mm]{"img/odroid"}
    \caption{ODROID-XU3 development board. Taken from \cite{ODROID_XU3}}
    \label{fig:odroid}
\end{figure}


The ODROID-XU3 processor has four Samsung Exynos-5422 Cortex A15 and four Cortex A7 cores,
with 2GB of LPDDR2 RAM. Only four cores are working at the same time and they are 
scheduled based on the big.LITTLE technology. The big.LITTLE technology
\cite{biglittle} automatically schedules workloads across cores based on
performance and energy needs. The vendor claims that the big.LITTLE technology
can achieve energy savings from 40\% to 
75\%, depending on the performance scenario \cite{biglittle}.
It is important to note that, even though the CPU contains eight cores, only four of then are working at a
given moment. The block diagram of the ODROID-XU3 can be seen in 
\ref{fig:odroidxu3-diagram}.


The ODROID-XU3 has a Texas Instrument power monitor chip (TI INA231) embedded
from origin. The TI INA231 provides an API to read the energy consumed by the 
cores and DRAM at a sampling rate of microseconds. These readings can be easily triggered and read
through software and consist of an accurate way to make fine-grained energy
consumption measurements. We assumed that the measurements made by the TI INA231 can 
be compared to the RAPL technology by Intel.


\begin{figure}[h!]
  \centering
    \includegraphics[width=\textwidth]{"img/odroidxu3-diagram"}
    \caption{ODROID-XU3 block diagram. Taken from \cite{ODROID_XU3}}
    \label{fig:odroidxu3-diagram}
\end{figure}


\subsection{Intel x86 architecture}

Across the different experiments, we have used three different machines running
on top of x86 Intel instruction sets to
compare with the ARM based machines. The Intel x86 machines are the most widely
used solutions for server and workstation applications.

The Intel Xeon that we used had RAPL enable (refer to Chapter X), which allowed us to measure energy
consumption accurately at a fine-grained level. Since the ATOM and QUAD machines
did not have RAPL technology enabled, we used a clamp power meter to measure the
energy consumed by the CPU at a given time.

The different types of measurements within the same architecture and
its possible affect on he final result are discussed in the Analysis section.

\textbf{describe more about Intel, its features (hyper threading, wich affects the results for example), its microarch families and where/how have them been used in production}

ex: "The Intel Atom is a brand name for a line of ultra-low-voltage CPUs by Intel. On the other hand, the x86 Intem Quad is brand name for a high performance family of Intel CPUs"

\clearpage


\section{Experiments setup}
\subsection{First set of experiments}

\subsubsection*{Hardware specifications}
For the first set of experiments, we used three machines with different hardware setups. The three machines differ in architecture and general purpose. The ARM\_virdis is a server rack with CPU consisting of ARMv7 processors produced by Boston Labs [ref]. We ran the same workloads in a x86 Intel Atom (Intel\_atom) and Intel Quad (Intel\_quad) for comparison. The Intel Atom is a brand name for a line of ultra-low-voltage CPUs by Intel. On the other hand, the x86 Intem Quad is brand name for a high performance family of Intel CPUs.

Below, we outline the most important specifications of the hardware setups we used for the experiments.

\vspace{10mm}
\textbf{Intel\_ATOM}
\begin{itemize}
  \item[] \textbf{kernel \& sys}:  Linux cern\-vm 2.6.32\-431.5.1.el6.x86\_64
  \item[] \textbf{OS}: Scientific Linux release 6.5 (Carbon)
  \item[] \textbf{CPU}:   4x Intel\textsuperscript{TM} Atom\textsuperscript{TM} CPU D525 \@ 1.8GHz
  \item[] \textbf{Memory  (MemTotal)}:        3925084  kB (4GB)
\end{itemize}

For more detailed specs refer to \cite{atom_specs}

\vspace{10mm}
\textbf{Intel\_QUAD}
\begin{itemize}
  \item[] \textbf{kernel \& sys}:  Linux cern-vm 2.6.32-431.5.1.el6.x86\_64
  \item[] \textbf{OS}: Scientific Linux release 6.5 (Carbon)
  \item[] \textbf{CPU}:    4x Intel\textsuperscript{TM} Core\textsuperscript{TM}2 Quad CPU  Q9400  \@ 2.66GHz
  \item[] \textbf{Memory  (MemTotal)}:        7928892 kB (8GB)
\end{itemize}

For more detailed specs refer to \cite{quad_specs}

\vspace{10mm}
\textbf{ARM\_Viridis}
\begin{itemize}
  \item[] \textbf{kernel \& sys}:  Linux  3.6.10-8.fc18.armv7hl.highbank
  \item[] \textbf{OS}: Fedora release 18 (Spherical Cow)
  \item[] \textbf{CPU}:    4x Quad-Core ARM\textsuperscript{TM} CortexA9\textsuperscript{TM} processor \@1.4GHz
  \item[] \textbf{Memory  (MemTotal)}:        4137780 kB (4GB) 
\end{itemize}

For more detailed specs refer to \cite{viridis_specs}


\subsubsection*{Software and workload}
We used the CMSSW framework in the generation-simulation mode (GEN\-SIM). The workflow performs a Monte Carlo simulation of 8 TeV LHC Minimum bias event using Pynthia8 (generation step), followed by Simulation with Geant4 (simulation step). For more information about the CMSSW framework and its limitation on ARM, refer to Chapter X.
At the time of the experiments, the CMSSW port for ARM had limitations on the multithreading support. We wanted to study the energy consumption of each hardware setup given different core load. Thus, we spinned up different processes instead of threads. The core-load levels used were 1/4, 1/2, 1 and 2 processes per number of physical cores. 

\subsubsection*{Metrics}
The energy efficiency metric used in this study is the ratio of performance per power consumed (in Watts). Performance consist on the average of events computed per second. Considering this metrics for comparing energy consumption, we consider a system to be as energy efficient as higher the ratio \textit{nr\_of\_events/s/W} is.

Given the hardware disparities of the setups we had in place to run our experiments, we used the performance (average fd events computed per second) as a way to uniform the results.

\subsubsection*{Tools for measuring energy consumption}
For this set of experiments, we performed physical measurements using an external clamp meter. The clamp was a Mini AC/DC Clamp meter
Mastech MS2102 AC/DC (see Figure \ref{fig:clamp}). The clamp meter supports a maximum of 200A current, which was enough for our experiments. In addition, it presents an accuracy of +-2.5\%. For more specifications about the clamp used, refer to [REF].

\begin{figure}[]
  \centering
    \includegraphics[scale=2.5]{"img/clamp"}
    \caption{Mastech MS2102 clamp meter used to measure energy consumption. Taken from \cite{} - cite Mastech website}
    \label{fig:clamp}
\end{figure}


% Please add the following required packages to your document preamble:
% \usepackage{graphicx}
\begin{table}[]
\centering
\resizebox{\textwidth}{!}{%
\begin{tabular}{|c|c|c|c|c|c|}
\hline
\textbf{\begin{tabular}[c]{@{}c@{}}Machine \\ codename\end{tabular}} & \textbf{Architecture} & \textbf{CPU} & \textbf{\begin{tabular}[c]{@{}c@{}}N\textsuperscript{o} active \\ cores\end{tabular}}  & \textbf{RAM} & \textbf{Notes} \\ \hline
\textbf{ARM\_viridis} & \begin{tabular}[c]{@{}c@{}}Quad-Core ARM\textsuperscript{TM} \\ CortexA9\textsuperscript{TM}\end{tabular} & ARMv7 32b (A7) & 4 & 2 GB & \begin{tabular}[c]{@{}c@{}}Server class ARM \\ processor with \\ ipmitools\end{tabular} \\ \hline
\textbf{Intel\_ATOM} & Intel Bonnell\textsuperscript{TM} & Atom D525 & 4 & 4GB & \begin{tabular}[c]{@{}c@{}}No internal \\ measurement \\ tool\end{tabular} \\ \hline
\textbf{Intel\_QUAD} & \begin{tabular}[c]{@{}c@{}}Intel Sandy \\ Bridge\textsuperscript{TM} \end{tabular} & \begin{tabular}[c]{@{}c@{}}Quad CPU Q9400\end{tabular} & 4 & 8GB & \begin{tabular}[c]{@{}c@{}}No internal\\  measurement \\ tool\end{tabular} \\ \hline
\end{tabular}
}
\caption{Summary of the 1SE specifications}
\label{my-label}
\end{table}


\subsection{Second set of experiments}

\subsubsection*{Hardware specifications}

For the second set of experiements, we again used three machines with different hardware setups. As in the 1SE, the three machines differ in architecture and general purpose. The ARM\_virdis, which was used in the 1SE, was also used during the second set of experiments. In addition to ARM\_viridis, we also resort to another machine powered by an ARM CPU. The ARM\_odroid is a development board manufactured by HardKernel [ref] and it is intended to provide a cheap and easy way to develop hardware and software in a ARM architecture. To represent the Intel architecture we used Intel\_xeon, a machine from the Intel Sandy Bridge family and powered by an Intel R5-2650 CPU. This machine was part of a server rack and it was intended for high performace scientific computation in a production scenario.

Below, we outline the most important aspects of the hardware setups we used for the experiments.

\vspace{10mm}
\textbf{ARM\_Viridis}
\begin{itemize}
  \item[] \textbf{kernel \& sys}:  Linux  3.6.10-8.fc18.armv7hl.highbank
  \item[] \textbf{OS}: Fedora release 18 (Spherical Cow)
  \item[] \textbf{CPU}:    4x Quad-Core ARM\textsuperscript{TM} CortexA9\textsuperscript{TM} processor \@1.4GHz
  \item[] \textbf{Memory  (MemTotal)}:        4137780 kB (4GB) 
\end{itemize}

For more detailed specs refer to \cite{viridis_specs}


\vspace{10mm}
\textbf{ARM\_odroid}
\begin{itemize}
  \item[] \textbf{kernel \& sys}:  Linux 3.10.24 LTS
  \item[] \textbf{OS}: Ubuntu 14.04.3 LTS (Trusty Tahr)
  \item[] \textbf{CPU}: 2x A15 and/or A7 cores(big.LITTLE technology) - A7 at 1.4GHz and A15 at 2GHz
  \item[] \textbf{Memory}:        2GB
\end{itemize}

For more detailed specs refer to \cite{odroid_specs}


\vspace{10mm}
\textbf{Intel\_xeon}
\begin{itemize}
  \item[] \textbf{kernel \& sys}:  Linux cern-vm 2.6.32-431.5.1.el6.x86\_64
  \item[] \textbf{OS}: Scientific Linux release 6.5 (Carbon)
  \item[] \textbf{CPU}:    4x Intel\textsuperscript{TM} CPU E5-2650 \@2GHz
  \item[] \textbf{Memory}:        252GB
\end{itemize}

For more detailed specs refer to \cite{xeon_specs}


\subsubsection*{Software and workload}
We used the CMSSW's mode ParCullCMS for generating the workload. The ParFullCMS mode is a multi-threaded Geant4 \cite{GEANT4} benchmark. It uses a complex CMS geometry for the event simulation and has the advantage of being multithreaded in both Intel and ARM architectures. As in the first set of experiements, we measured the energy consumed by the machine under different physical core loads. The core-load levels used were 1/4, 1/2, 1 and 2 threads per number of physical cores. 

\subsubsection*{Metrics}
As in the 1SE, the energy efficiency metric used in this study is the ratio of performance per power consumed (Watts). The hardware setups used in the 2SE differ in specs and features. Therefore, we used the this metric as a way to uniform the results.

\subsubsection*{Tools for measuring energy consumption}
For the 2SE, we performed both internal and external measurements in the Intel\_xeon and ARM\_odroid. On the ARM\_viridis, we performed only internal measurements given the lack of a tool that would performe with the same degree of accuracy than the tools used for Intel\_xeon and ARM\_odroid. All the tools used to measure energy consumption were embeeded in the hardware setup of the machines.

For the ARM\_odroid, we used a Texas Instrument power monitor chip (TI INA231) for internal measurements. The TI INA231 allowed us to sample the energy consumed by the cores and DRAM at a frequency rate of microseconds. For the extrenal measurements on the ARM\_odroid, we used an external plug-in power monitor with a computer interface for sampling and storing the results.

For the Intel\_xeon machine, we used the Running Average Power Unit (RAPL) technolgy to perform internal measurements. The RAPL allowed us to sample the energy consumed by the CPU's package, DRAM and cores. For the external measurements, we used an API provided by the server rack's PDU. This API provides a measure sampling rate of around 1 second. 

For the ARM\_viridis, we used the capabilities of the Intellegent Platform Management Interface (IPMI) \cite{IPMI} included in the server from origin. The IPMI is a chip that runs as a separate subsystem and is attached to the motherboard. The ARM\_viridis implementation of IPMI provide several capabilities, namely interlal hardware energy monitoring. We leveraged the IPMI tools to perform internal energy consumption of the ARM cores during the experiments



% Please add the following required packages to your document preamble:
% \usepackage{graphicx}
\begin{table}[]
\centering
\resizebox{\textwidth}{!}{%
\begin{tabular}{|c|c|c|c|c|c|}
\hline
\textbf{\begin{tabular}[c]{@{}c@{}}Machine \\ codename\end{tabular}} & \textbf{Architecture} & \textbf{CPU} & \textbf{\begin{tabular}[c]{@{}c@{}}N\textsuperscript{o} active \\ cores\end{tabular}} & \textbf{RAM} & \textbf{Notes} \\ \hline
\textbf{ARM\_odroid} & \begin{tabular}[c]{@{}c@{}}Quad-Core \\ ARMv7\textsuperscript{TM}\end{tabular} & \begin{tabular}[c]{@{}c@{}}A15 and or A7 \\ cores(big.LITTLE technology)\end{tabular} & 4 & 2 GB & \begin{tabular}[c]{@{}c@{}}Development \\ board with TI \\ INA231 chip\end{tabular} \\ \hline
\textbf{ARM\_viridis} & \begin{tabular}[c]{@{}c@{}}Quad-Core \\ ARM\textsuperscript{TM} CortexA9\textsuperscript{TM}\end{tabular} & ARMv7 32b (A7) & 4 & 2 GB & \begin{tabular}[c]{@{}c@{}}Server class ARM \\ processor with \\ ipmitools\end{tabular} \\ \hline
\textbf{Intel\_xeon} & \begin{tabular}[c]{@{}c@{}}Intel Sandy\\ Bridge\textsuperscript{TM}\end{tabular} & CPU E5-2650 & 32 & 252 GB & \begin{tabular}[c]{@{}c@{}}System on \\a rack  with\\ RAPL\end{tabular} \\ \hline
\end{tabular}
}
\caption{Summary of the 2-SE specifications}
\label{my-label}
\end{table}



\clearpage 

\section{Summary}

We have performed several experiments under different hardware setups. 
Our main goal was to understand how the ARM and Intel architectures
perform under similar workloads from an energy consumption standpoint.

In this chapter we outlined the setup of the machines used during the experiments.

The hardware setups were chosen given their similarity and possibility of a reliable comparison and hardware availability. It is important to note that both ARM\_viridis and ARM\_odroid machines are much more recent than the compared Intel
hardware. All the ARM machines used were still a technology that was yet to find production
stability at the moment of the experiments. On the other hand, the Intel architecture used in this study was widely used in real HPC applications at the time of this study.

For this study, we assume that the RAPL, the internal TI INA231 chip and the IPMI tools for internal energy
consumption measurement are similarly accurate and would produce the same results if interchanged. 

